\documentclass{article}
\usepackage{graphicx}
\usepackage{tabularx}
\usepackage{fancyhdr}
\usepackage{cite}
\usepackage{hyperref}
\usepackage{lipsum}
\usepackage{xcolor}

\pagestyle{fancy}
\setlength{\footskip}{60pt}
\vspace{-2cm}
\fancyfoot{\rule{0.5\linewidth}{1pt}}
\fancyfoot[L]{{\includegraphics[scale=0.10]{logo.png}}}

\begin{document}


\begin{center}
\includegraphics[scale=0.2]{logo.png}
\end{center}

\vspace{0.5cm}

\begin{center}

 Team's name: FamilyName1, FamilyName2, FamilyName3, FamilyName4 \\

\vspace{0.5cm}

{\Huge \textcolor{red}{Recovery and stress meter}} \\


\vspace{0.5cm}

{\Large \textbf{Project plan}} \\


\end{center}

\vspace{1cm}

\begin{center}

{\Large }



\begin{tabular}{l}
 First year hardware project \\
 School of ICT\\
 Metropolia University of Applied Sciences  \\
 \today \, (v0.2)
\end{tabular}
\end{center}

\newpage

\begin{abstract}
Write your abstract here.
\end{abstract}




\begin{center}
\textbf{\large Keywords:}
\end{center}

\vspace{1cm}

\begin{table}[h]
\centering
\begin{tabular}{|c|c|c|c|}
\hline
\textbf{Ver} & \textbf{Description} & \textbf{Date} & \textbf{Author(s)} \\ \hline
0.1 & First version created from the Construx's software development plan template. & 9.12.2022 & Sakari Lukkarinen \\ \hline
0.2 & Continued editing. Added examples and other documents. & 11.12.2022 & Sakari Lukkarinen \\ \hline
\end{tabular}
\caption{Version history}
\label{table:version-history}
\end{table}

\begin{bf}
\begin{center}
\fbox{
\parbox{0.8\textwidth}{
Keyword 1\\ Keyword 2 \\ Keyword 3 \\
...
}
}
\end{center}
\end{bf}

\newpage

\tableofcontents

\section{Introduction}
This section describes the recovery and stress meter project being to be
conduct during the first year Hardware courses of School of ICT at Metropolia
University of Applied Sciences.



\subsection{Project Overview and Vision}
The objective of the project is to create a proof-of-concept product of a stress and
recovery meter. The project begins with the discovery phase, where the
requirement specification is reviewed, the project plan is developed with detailed
schedule and risk management plans. The planning checkpoint review is
arranged at the beginning of the 4th learning period starting in the middle of
March.

In the 4th learning period, the project continues to the invention phase where the
architecture of the system and detailed design are planned. The actual
implementation work is divided into three (3) iterative development stages each
of them having a separate release and milestones. The final release is at the end
of the spring semester.

During the development the project plan with time schedules and risk
management are updated under change and version control.


\subsection{Project Deliverables}
\subsection{Evolution of the Project Management Plan}
\subsection{Reference Materials}
\subsection{Definitions and Acronyms}

\section{Project Organization}
A staged delivery plan is followed having at least 3 stages and the final release.
The project’s process model, organizational structure (chain of command and
management reporting structure), and responsibilities of individuals on the project
are described in this chapter.



\subsection{Process Model}
A staged delivery plan is followed in the project. A detailed GANTT chart of major
phases and milestones are given in a separate document \textcolor{red}{(XXX. Project Phases
and Milestones)}. The following Table 3 summarises the major work products,
their planned completion dates and their content.

[table]

\textcolor{red}{Instructions: consider including all the top-level work products.}

\subsection{Organizational Structure and Project Responsibilities}

\section{Managerial Process}
The management objectives, priorities, project assumptions, dependencies,
constraints,
risk
management
techniques,
monitoring
and
controlling
mechanisms, and the staffing plan are described here.

\subsection{Management Objectives and Priorities}
\subsection{Assumptions, Dependencies, and Constraints}
\subsection{Risk Management}
\subsection{Monitoring and Controlling Mechanisms}
\subsection{Staffing Plan}

\section{Technical Process}
\subsection{Methods, Tools and Techniques}
\subsection{Documentation}
\subsection{Project Support Functions (optional)}

\section{Work Packages, Schedule, and Budget}
\subsection{Work Packages and their dependencies}
\subsection{Resource Requirements}
\subsection{Time Budget and Schedule}

\section{Additional Components (optional)}

\end{document}
