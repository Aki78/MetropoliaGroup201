\documentclass{article}
\usepackage{graphicx}
\usepackage{tabularx}
\usepackage{fancyhdr}
\usepackage{cite}
\usepackage{hyperref}
\usepackage{lipsum}
\usepackage{xcolor}
\usepackage{colortbl}

\pagestyle{fancy}
\setlength{\footskip}{60pt}
\vspace{-2cm}
\fancyfoot{\rule{0.5\linewidth}{1pt}}
\fancyfoot[L]{{\includegraphics[scale=0.10]{logo.png}}}

\definecolor{lightgreen}{rgb}{0.56, 0.93, 0.56}

\begin{document}


\begin{center}
\includegraphics[scale=0.2]{logo.png}

\end{center}

\vspace{0.5cm}

\begin{center}

 Team's name: FamilyName1, FamilyName2, FamilyName3, FamilyName4 \\

\vspace{0.5cm}

{\Huge \textcolor{red}{Recovery and stress meter}} \\


\vspace{0.5cm}

{\Large \textbf{Requirement Specifications}} \\


\end{center}

\vspace{1cm}

\begin{center}

{\Large }



\begin{tabular}{l}
 First year hardware project \\
 School of ICT\\
 Metropolia University of Applied Sciences  \\
 \today \, (v1.1)
\end{tabular}
\end{center}

\newpage

\begin{abstract}
Write your abstract here.
\end{abstract}





\begin{table}[h]
\centering
\caption{Version history}
\label{tab:version_history}
\begin{tabular}{|c|p{5cm}|c|l|}
\hline
\textbf{Ver} & \textbf{Description} & \textbf{Date} & \textbf{Author(s)} \\
\hline
0.1 & First draft. Translated from Finnish template. & 23.9.2022 & SL \\
\hline
0.2 & Working principle added to the introduction. Description of the current situation added. & 24.11.2022 & SL \\
\hline
0.3 & Description of the target state added. & 25.11.2022 & SL \\
\hline
0.4 & Continuing description of the target state. & 28.11.2022 & SL \\
\hline
0.5 & Added heart rate detection and heart rate variability to Ch 3. Edited Ch 4. Added key features, use and users, and platforms of application. & 2.12.2022 & SL \\
\hline
0.6 & Added first draft of functional and non-functional requirements, users and use cases & 3.12.2022 & SL \\
\hline
0.7 & Updated non-functional requirements. Minor changes and proof reading. & 11.12.2022 & SL \\
\hline
0.8 & Some language corrections up to 3.3. May need further correcting. & 13.12.2022 & SH, SL \\
\hline
0.9 & Final reviews before release to workspaces & 19.12.2022 & SL \\
\hline
1.0 & Added Table 3. Components & 23.1.2023 & SL \\
\hline
1.1 & Some minor language corrections and updates & 25.1.2023 & S \\
\hline
\end{tabular}
\end{table}

\pagebreak

\tableofcontents

\newpage

\section{Introduction}
This document is part of a hardware project for the first year ICT engineering students
studying at Metropolia University of Applied Sciences. The aim of this document is to give
the requirement specifications for a new health technology device for measuring recovery
and stress index using optically detected heart rate and its variability.
The state of the autonomous nervous system (ANS) can be estimated from the heart rate
variation. Nowadays most of the wearable activity tracking devices and sports watches
detect the heart rate and its variability either electrically (e.g. detecting the electrocardiogram
or some parts of its signal) or optically (e.g. optical heart rate detector or oxygen saturation
detectors). \ref{cleveland_clinic_2021}
Heart rate variability (HRV) is an accurate method to assess the autonomic nervous system
(ANS) function. HRV is widely used by health and wellbeing professionals to objectively
measure the physiological and mental stress and recovery. In addition, HRV is a commonly
used tool in the research of different cardiovascular and metabolic diseases and their risk
factors. [2]
The aim for the project is to build an objective and easy to use device for measuring HRV
and estimate the current stress or recovery status. The device is intended to be used in
home or office environments either by the end users themselves or together with health and
wellbeing professionals such as physiotherapists, nurses or medical doctors.
The device detects the heart rate and its variability using a photoplethysmography (PPG). It
measures optically blood volume changes in the microvascular bed of tissue. The change in
volume is detected by measuring the light emitted by the light emitting diodes (LEDs),
absorbed by the tissues and detected with photodiodes. The heart rate can be measured
from the peaks of the alternating signal presenting the volumetric blood changes in the
tissue. [3]

\pagebreak

\section{Concepts and definitions}
\begin{table}[h]
\label{tab:deliverable}
\resizebox{0.75\columnwidth}{!}{
\begin{tabular}{p{2cm} p{9cm}}
$\alpha$ & Slope of the linear interpolation of the spectrum in a log-log scale \\
ANS & Autonomous nervous system \\
BPM & Beat per minute \\
ECG & Electrocardiogram \\
IBI & Inter-beat-interval, measured from PPG signal, given in milliseconds (ms) \\
HF & High frequency \\
HR & Heart rate, typically given in units of beat per minute (BPM) \\
HRV & Heart rate variability, measures how much there is variability in the heart rate from  beat to beat over longer time period, can be characterized by several parameters  \\
LAN & Local area network \\
LED & Light emitting diode \\
LF & Low frequency \\
LF/HF & Ratio of LF/HF \\
NN  interval & Time difference between two peaks either in ECG or PPG signal, either PPI and RRI \\
NN50 count & Number  of pairs of adjacent NN intervals differing more than 50 ms \\
OLED & Organic light emitting diode \\
pNN50 & NN50 count divided by the total number of all NN intervals \\
PPI & peak-to-peak interval, time difference between two pulse peaks in photoplethysmography signal \\
PPG & Photoplethysmography, optically detected heart pulse typically detected from peripheral blood circulation, like from finger, wrist, toe, or ear lobe \\
PNS & Parasympathetic nervous system, part of autonomic nervous system \\
PTSD & Post-traumatic stress disorder \\
RMSSD & The square root of sum of squares of differences (ms) \\
RRI & RR-interval, time difference between two R-peaks in ECG signal \\
SD1 & Poincaré plot index \\
SD2 & Poincaré plot index \\
SDANN & Standard deviation of the average of NN intervals (ms) \\
SDNN & Standard deviation of all NN intervals (ms) \\
SI & Baevsky’s stress index \\
SDSD & Standard deviation of differences between adjacent NN intervals (ms) \\
SNS & Sympathetic nervous system, part of autonomic nervous system \\
ULF & Ultra-low frequency \\
USB & Universal serial port \\
WiFi & Wireless fidelity \\
VLF & Very flow frequency \\
\end{tabular}
}
\end{table}

\pagebreak

\section{Description of the current situation}
\subsection{Stress}
Stress is defined as “a physical, mental, or emotional factor that causes bodily or mental
tension” [4]. According to the American Institute of Stress [5] [6]:

\begin{itemize}
\item 77 \% of people experience stress that affects their physical health
\item 73 \% of people have stress that impacts their mental health
\item 48 \% of people have trouble sleeping because of stress
\item 33 \% of people report feeling extreme stres
\end{itemize}

\begin{figure*}[h]
  \centering
  \includegraphics[width=0.7\textwidth]{sick.png}
  \caption{ Example of project's organisational chart (Source: Harrin, 2017).}
  \label{harrin}
\end{figure*}

Physiological or mental imbalance can induce stress. Our autonomic nervous system (ANS)
quickly responds with physiological changes through our sympathetic (SNS) and
parasympathetic (PNS) nervous systems. During the stress response our body’s endocrine
system releases hormones, and several changes in our physiological state occur. For
example, heart rate (HR) can even double or triple and causes changes to HRV. [8]

\subsection{Detecting heart rate}
The heart rate or pulse rate measures how often the heart beats and is given units of beats
per minute (BPM). Usually, the heart rate varies on the body’s physical need, but is also
affected by physical fitness, stress of psychological status, diet, drugs, hormones,
environment and diseases and illnesses. The normal resting adult heart rate is 60-100 BPM.
During sleep, a heart rate of 40-50 BPM is common and considered normal. [9]

4 (16)
Heart rate variability (HRV) is the variation of the time intervals between heartbeats and it is
measured in units of seconds, or more commonly, in milliseconds (ms). Other terms used
include RR interval (RRI) variability, where R corresponds to the peak of QRS-complex of
electrocardiography (ECG), and Peak-to-Peak interval, if the HRV is measured optically.
Figure 2 visualizes heart HRV with R-R interval changes. [10]


\begin{figure*}[h]
  \centering
  \includegraphics[width=0.7\textwidth]{heart_rate.png}
  \caption{ Heart rate variability (HRV) calculated from the R-R intervals (RRI) [10].}
  \label{harrin}
\end{figure*}


Heart rate variability can be detected with various methods. ECG is considered the golden
standard for HRV measurement [10]. Other methods are photoplethysmography (PPG),
which detects the heart rate variability optically, usually measured from fingers, wrists,
forehead or earlobes, blood pressure or ballistocardiography, which measures small
changes in body’s weight when the blood flows from the heart to the aorta.

Figure 3 shows a typical fitness and wellness watch having an optical heart rate sensor [11].
The light emitting diodes (LEDs) and optical detectors are seen on the back of the watch.

\begin{figure}[h]
\begin{minipage}{0.5\textwidth}
\includegraphics[width=\linewidth]{watch1.png}
\label{fig:sub1}
\end{minipage}
\begin{minipage}{0.5\textwidth}
\includegraphics[width=\linewidth]{watch2.png}
\label{fig:sub2}
\end{minipage}
\caption{ An example of fitness and wellness watch having an optical heart rate sensor. [11]}
\label{fig:test}
\end{figure}



Figure 4 shows an example of photoplethysmography signal recorded with wrist worn pulse
oximetry [12]. The device is shown on the left. The sensor is attached to the thumb. The
PPG signal is shown on the right. The inter-beat-interval (IBI) is calculated from the negative
peaks (the bottoms) of the PPG signal. It could be calculated also from the positive peaks
(the maximum) or from the rising edges of the signal.


\begin{figure}[h]
\begin{minipage}{0.5\textwidth}
\includegraphics[width=\linewidth]{hand.png}
\label{fig:sub1}
\end{minipage}
\begin{minipage}{0.5\textwidth}
\includegraphics[width=\linewidth]{ppg_signal.png}
\label{fig:sub2}
\end{minipage}
\caption{Caption for both figures}
\label{fig:test}
\end{figure}

\subsection{Heart rate variability}
At present there is no accepted standard for stress evaluation. However, several HRV
variables change in response to stress. Usually, stress induces low parasympathetic
nervous system activity, which is associated with variation of some HRV variables such as
high-frequency band and an increase in the low-frequency band. [8]
European Society of Cardiology together with the North American Society of Pacing and
Electrophysiology have defined and established the standards for the measurement,
physiological interpretation, and clinical use of HRV [13]. Most used selected time-domain
and frequency-domain measures of HRV are summarized in Table 1 and Table 2.


\begin{table}[h]
\centering
\caption{Selected time-domain measures of HRV [13]}
\label{tab:HRV}
\begin{tabular}{llp{7cm}}
\hline
\textbf{Variable} & \textbf{Units} & \textbf{Description} \\ \hline
SDNN & ms & Standard deviation of all NN intervals \\ \hline
SDANN & ms & Standard deviation of the averages of NN intervals in all 5-minute segments of the entire recording \\ \hline
RMSSD & ms & The square root of the mean of the sum of the squares of differences between adjacent NN intervals \\ \hline
SDNN index & ms & Mean of the standard deviations of all NN intervals for all 5-minute segments of the entire recording \\ \hline
SDSD & ms & Standard deviation of differences between adjacent NN intervals \\ \hline
NN50 count & ms & Number of pairs of adjacent NN intervals differing by more than 50 ms in the entire recording \\ \hline
pNN50\% & - & NN50 count divided by the total number of all NN intervals \\ \hline
HRV triangular index & ms & Total number of all NN intervals divided by the height of the histogram of all NN intervals \\ \hline
TINN & ms & Baseline width of the minimum square difference triangular interpolation of the highest peak of the histogram of all NN intervals \\ \hline
Differential index & ms & Difference between the widths of the histogram of differences between adjacent NN intervals measured at selected heights \\ \hline
Logarithmic index & - & Coefficient $\phi$ of the negative exponential curve $k · e^{-\phi t}$, which is the best approximation of the histogram of absolute differences between adjacent NN intervals \\ \hline
\end{tabular}
\end{table}



\begin{table}[h]
\centering
\caption{Selected frequency-domain measures of HRV [13]}
\label{tab:HRV_freq}
\begin{tabular}{|l|l|l|}
\hline
\textbf{Variable} & \textbf{Units} & \textbf{Description} \\ \hline
\multicolumn{3}{|c|}{\textbf{Analysis of short-term recordings (5 min)}} \\ \hline
5min total power & ms$^2$ & The variance of NN intervals over the temporal segment \\ \hline
VLF & ms$^2$ & Power in VLF range (buffer 0.4 Hz) \\ \hline
LF & ms$^2$ & Power in LF range (0.04–0.15 Hz) \\ \hline
LF norm & n.u. & LF power in normalized units (LF/(total power-VLF)×100) \\ \hline
HF & ms$^2$ & Power in HF range (0.15–0.4 Hz) \\ \hline
HF norm & n.u. & HF power in normalized units (HF/(total power-VLF)×100) \\ \hline
LF/HF & - & Ratio LF/HF \\ \hline
\multicolumn{3}{|c|}{\textbf{Analysis of entire 24 hours}} \\ \hline
Total power & ms$^2$ & Variance of all NN intervals \\ \hline
ULF & ms$^2$ & Power in the ULF range (buffer 0.003 Hz) \\ \hline
VLF & ms$^2$ & Power in the VLF range (0.003–0.04 Hz) \\ \hline
LF & ms$^2$ & Power in the LF range (0.04–0.15 Hz) \\ \hline
HF & ms$^2$ & Power in the HF range (0.15–0.4 Hz) \\ \hline
$\alpha$ & ms$^2$ & Slope of the linear interpolation of the spectrum in a log-log scale \\ \hline
\end{tabular}
\end{table}

\subsection{Recovery and stress indexes}
Based on the common measures of HRV, special indexes representing the parasympathetic
and sympathetic cardiac activity have been developed. For example, Kubios HRV software
is based on the following parameters to calculate PNS and SNS indexes [14]:


\begin{itemize}
\item Mean RR interval
\item Root mean square of successive RR interval differences (RMSSD)
\item Poincaré plot index SD1 and SD2 in normalized units
\item Baevsky’s stress index (SI)
\end{itemize}

abc

Each parameter is compared to their normal population values and the values are then
scaled with standard deviations (SD) of normal population and finally a proprietary weighting
is applied to obtain the index values. These are illustrated in Figure 5 and Figure 6.


\begin{figure*}[h]
  \centering
  \includegraphics[width=0.7\textwidth]{parasympathetic.png}
  \caption{Parasympathetic nervous system (PNS) index. High positive values are interpreted as a good recovery
of the test subject. Source: [14]}
  \label{harrin}
\end{figure*}

\begin{figure*}[h]
  \centering
  \includegraphics[width=0.7\textwidth]{sympathetic2.png}
  \caption{Sympathetic nervous system (SNS) index. High negative values are interpreted as a low stress of the
test subject. Source: [14]}
  \label{harrin}
\end{figure*}

\section{Description of the target state}

\subsection{The purpose}
The aim is to develop a working proof-of-concept of the recovery and stress meter. A
suitable microcontroller board and additional components are used. The Raspberry Pi Pico
was selected for that purpose, as the Raspberry Pi products are extensively supported by
the manufacturers and by the user community.
Metropolia University of Applied Sciences’ teaching personnel together with senior students
have evaluated and selected the hardware components for the project. In addition, a special
board for development of IoT devices with Raspberry Pi Pico is designed and tested by one
of the senior lecturers. The background development and research results are openly
available and readable in Theseus [16] [17] [18].
ABC


\begin{figure*}[h]
  \centering
  \includegraphics[width=0.7\textwidth]{raspi_heart.png}
  \caption{Photograph of the development board with connected Raspberry Pi Pico board, OLED display and
optical heart rate sensor.}
  \label{harrin}
\end{figure*}

ABC

\subsection{Application concept}
The core of the proof-of-concept is Raspberry Pi Pico, a small and versatile microcontroller
board designed for IoT devices. The device is adaptable to a wide range of applications in
home, hobby, education, and industry. It is programmable both in C and MicroPython, of
which MicroPython is used for this project. The device has a rich set of peripherals, including
SPI, I2C, and programmable I/O state machines for custom peripheral support. It has also a
wireless version having a fully certified wireless LAN module. [19]

ABC

The development board is shown in Figure 7. On the bottom left the rotary switch
and knob is shown. Above it is a 128x64 wide OLED display. At the upper left corner 3 LEDs
and two of the three micro buttons are shown, which can be used for interacting with the
development board. In the middle, the Raspberry Pi Pico board with soldered pins is shown.


The Pico board is connected to the laptop or desktop through a USB-cable (black cable at
the top of the figure). On the right, 4-pin Grove-connectors for connecting serial
communication devices, like I2C sensors or analog input sensors, are shown. The optical
heart rate sensor is connected to an ADC\_0 pin and is shown above the development board.
The components used in the proof-of-concept product are listed in Table 3.


\begin{table}[h]
\centering
\caption{Component used in the proof-of-concept product.}
\label{tab:POC_components}
\begin{tabular}{|l|p{5cm}|}
\hline
\textbf{Component} & \textbf{Description} \\ \hline
Raspberry Pi & Dual-core ARM processor microcontroller having 246 kB SRAM and 2 MB on-board Flash. It also includes 2.4 GHz wireless LAN and 26 multifunction GPIO pins.\\ \hline
Crowtail Pulse Sensor v2.0 & Optical heart rate sensor with LED, photodiode, analog amplifier, and analog signal output. Operating voltage 3-5 V. \\ \hline
OLED Display & SSD1306 compatible 128x64 monochrome organic LED-display. Communicates with I2C or UART-protocol. \\ \hline
Protoboard & Passive protoboard specially designed for this project to help connect the other components to the Raspberry Pi. \\ \hline
Rotary knob & Digital rotary knob with push button. \\ \hline
\end{tabular}
\end{table}



\begin{table}[h]
\centering
\caption{Component used in the proof-of-concept product.}
\label{tab:POC_components}
\begin{tabular}{|l|p{5cm}|l|}
\hline
\textbf{Component} & \textbf{Description} & \textbf{More Info} \\ \hline
Raspberry Pi & Dual-core ARM processor microcontroller having 246 kB SRAM and 2 MB on-board Flash. It also includes 2.4 GHz wireless LAN and 26 multifunction GPIO pins. & Raspberry Pi website \\ \hline
Crowtail Pulse Sensor v2.0 & Optical heart rate sensor with LED, photodiode, analog amplifier, and analog signal output. Operating voltage 3-5 V. & Elecrow website \\ \hline
OLED Display & SSD1306 compatible 128x64 monochrome organic LED-display. Communicates with I2C or UART-protocol. & SSD1306 OLED Display datasheet \\ \hline
Protoboard & Passive protoboard specially designed for this project to help connect the other components to the Raspberry Pi. & N/A \\ \hline
Rotary knob & Digital rotary knob with push button. & N/A \\ \hline
\end{tabular}
\end{table}

\subsection{Operating principle}

The heart rate is detected using the optical heart rate sensor (Pulse Sensor v2.0, Crowtail)
[20]. The analog signal is converted into digital using Raspberry Pi Pico’s AD-converters.
The heart rate is calculated using peak-detection algorithms for photoplethysmography
(PPG) signals using Pico’s central processing unit (CPU). The operation can be controlled
using the rotary switch and knob. Results and feedback to the user are shown on the OLED
display. In addition, the extra LEDs can be used to indicate, for example, the quality of the
signal or the data collection operations to the user.
The data is preprocessed with Pico. Pico’s wireless connection can be used to send the data
to a cloud server and return the analysis results to the development board and show them to
the user. Figure 88 illustrates the cloud and web-server architecture of the system.

ABC
\begin{figure*}[h]
  \centering
  \includegraphics[width=0.7\textwidth]{web_server.png}
  \caption{ Example of project's organisational chart (Source: Harrin, 2017).}
  \label{harrin}
\end{figure*}

ABC

The development board acquires and records the optical heart pulse data and preprocesses
the data as seen in Figure 8. All heart rate calculations can be also processed in the
development board. The time stamped preprocessed pulse data is sent wirelessly to the
base station which sends the data to the Web server. The data is stored in the webserver’s
database where more analysis and reporting can be done. The health care professional
interacts with the webserver through Web clients.

\subsection{Key features}
The following Table 4. recaps the key features of the system.

ABC


\begin{table}
\begin{tabular}{|p{4cm}|p{7cm}|}
\hline
\textbf{Key Feature} & \textbf{Description} \\
\hline
HRV detection & PPG signal is detected using the optical pulse sensor and the heart rate variability is measured using the development board’s MCU. \\
\hline\
Display & The system has an OLED display capable of showing both text and graphics. \\
\hline\
Controls & The system has a rotary switch control knob with push button. The control knob can be used for controlling the operation of the system. \\
\hline
MCU & The system contains an MCU with Flash and RAM memory and several peripheral connections enabling to process the detected PPG signal and calculate the inter-peak-interval variations. \\
\hline
Wireless connection & The raw HRV data (PPI) can be further analysed using the system or sent wirelessly to a Cloud Server for further calculations. The system contains a wireless WiFi transmitter. \\
\hline
USB connection & The system contains a USB connection. The USB can be used to control, code, and download the executable files. In addition, it can be used to debug the code and download and upload data files. \\
\hline
\end{tabular}
\end{table}

\subsection{Use and users}
The final system is intended to be used for measuring the recovery and stress index based
on HRV analysis detected optically from the finger, wrist, hand palm, arm, upper arm, chest,
cheek, forehead, or earlobe.
The system is intended to be used by a person, patient, customer, or healthcare professional
aiming to measure the subject’s recovery and stress index. The information can be used to
help understand the study subject’s current situation. The system is used in a normal dry
home or office environment.

ABC

\subsection{Applications}

The system can be used to analyze the psychophysiological state. HRV is related to
emotional arousal, conditions of acute time pressure and emotional strain, and elevated
anxiety state. HRV has also been shown to be reduced in individuals reporting to worry
more. In individuals with post-traumatic stress disorder (PTSD), HRV is reduced. [10]

ABC


\section{Functional requirements}

ABC

The functionality of the device will be based on functional requirements, i.e. what the system
will contain and what it will not contain. This chapter presents the essential functional
requirements related to the system.
The original table is maintained in Excel and is attached to the document, making it easier to
structure the data. A snapshot of the table is shown here.

\begin{figure*}[h]
  \centering
  \includegraphics[width=0.7\textwidth]{screenshot.png}
  \caption{ Example of project's organisational chart (Source: Harrin, 2017).}
  \label{harrin}
\end{figure*}

ABC


\section{Non-functional requirements}

ABC

Non-functional requirements define the limitations and boundary conditions for functional
requirements. Non-functional requirements may relate to, for example, security and privacy,
scale, performance and response time, operating languages and localization, execution
environment, implementation techniques and languages, as well as compliance with
standards, usability, responsiveness, documentation, rights to implementation, and
customizability and accessibility of the implementation.
This chapter presents the essential non-functional requirements related to the system.
The original table is maintained in Excel and is attached to the document, making it easier to
structure the data. A snapshot of the table is shown here.
The minimal sampling rate of the PPG for accurate pulse rate variability parameters in
healthy volunteers is 50 Hz. For monitoring the average heart rate, 5 Hz sampling frequency
can be sufficient. Correct HRV analysis requires higher sampling rates. Interpolation can
improve HRV accuracy from lower temporal resolution PPGs. [21]


\begin{figure*}[h]
  \centering
  \includegraphics[width=0.7\textwidth]{screenshot2.png}
  \caption{ Example of project's organisational chart (Source: Harrin, 2017).}
  \label{harrin}
\end{figure*}


ABC


\section{Use cases}

ABC



ABC


\subsection{User roles}
Table X3 summarizes the different user roles.


\begin{table}[h]
\centering
\begin{tabular}{|c|c|p{4cm}|}
\hline
\textbf{User} & \textbf{Abbreviation} & \textbf{Description} \\
\hline
User & U1 & A person using the device \\
\hline
Medical professional & U2 & A medical doctor, nurse or other medical professional interpreting the results \\ \hline
System administrator & U3 & A technical person responsible for the system administration \\ \hline
Other role & Un & \\
\hline
\end{tabular}
\caption{User roles}
\label{tab:X3}
\end{table}

ABC

\subsection{Use cases and use case diagram}
Table X4 presents the application use cases and ties them to the previously presented user
roles. Table X4 also prioritizes (1=mandatory, 2=important, 3=useful) the use cases,
explains the interfaces to functional and non-functional requirements, and presents
dependencies on other use cases. For a more detailed description of the use cases, see
Section 7.3


ABC


\begin{table}[h]
\centering
\begin{tabular}{|c|p{3cm}|p{2cm}|p{2cm}|p{2cm}|p{2cm}|}
\hline
\textbf{ID} & \textbf{Name} & \textbf{User role(s)} & \textbf{Importance} & \textbf{Links to requirements} & \textbf{Links to use cases} \\
\hline
UC01 & Recording new HRV analysis & U1: User & 1 & FR01 & \\
\hline
UC02 & & & & & \\
\hline
UC03 & Reading previous HRV analysis & U1: User & 1 & FR02 & UC01 \\
\hline
\end{tabular}
\caption{Use cases}
\label{tab:my_label}
\end{table}


\subsection{Detailed description of the use cases}
Table X3 summarizes the different user roles.


ABC



\begin{table}[h]
\caption{Table X5. Use Cases.}
\label{table:X5}
\begin{tabular}{|l|l|}
\hline
\textbf{Use case ID} & UC01 \\
\textbf{Name} & Recording new HRV analysis \\
\textbf{Author and date} & Sakari Lukkarinen, 2.12.2022 \\
\textbf{User roles} & U1: User \\
\textbf{Importance} & 1 \\
\textbf{Links and sources} & FR01 \\
\hline
\end{tabular}
\end{table}

\textbf{Prerequisites:} The system is on and ready to record.

\textbf{Description:}
1. User selects new recording from the system.
2. User attaches the sensor on the skin.
3. User starts the recording. The system records the signal.
4. Recording period is over and the signal is analysed.
5. The results are shown on the display.

\textbf{Exceptions:}
1. The signal is low quality or the sensor is not properly on the skin. The user is warned about the situation and asked to restart recording.
2. After the recording the signal is too low quality. The user is warned about the situation and the results are not stored.

\textbf{Final result:} The HRV analysis is ready and shown to the user.

\textbf{Other requirements:} None.




































\bibliographystyle{unsrt}

\bibliography{references}


\end{document}
